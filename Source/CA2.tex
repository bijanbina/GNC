%----------------------------------------------------------------------------------------
%	PACKAGES AND OTHER DOCUMENT CONFIGURATIONS
%----------------------------------------------------------------------------------------

\documentclass{article}

\usepackage{fancyhdr} % Required for custom headers
\usepackage{hyperref} % Required for links
\usepackage{extramarks} % Required for headers and footers
\usepackage[usenames,dvipsnames]{color} % Required for custom colors
\usepackage{graphicx} % Required to insert images
\usepackage{listings} % Required for insertion of code
\usepackage{courier} % Required for the courier font
\usepackage[skip=0pt]{caption} % Required for figure captions
\usepackage{amsmath} % Required for equation in figure captions
\usepackage{xcolor}
\usepackage{color} %red, green, blue, yellow, cyan, magenta, black, white
\usepackage{changepage,titlesec} % Required for sectioning and margin
\usepackage{lastpage} % Required to determine the last page for the footer
\usepackage{amssymb} % Used for math sign use
%PineGreen

%equation contains section
\numberwithin{equation}{section}  

%PineGreen
\definecolor{ao(english)}{rgb}{0.0, 0.5, 0.0}
\definecolor{airforceblue}{rgb}{0.27, 0.47, 0.6}
\definecolor{Headblue}{rgb}{0.25, 0.37, 0.5}
\definecolor{Linkblue}{rgb}{0.20, 0.4, 0.73}
\newcommand{\hwMaincolor}{black}
\newcommand{\hwRulecolor}{black}
\def\chpcolor{airforceblue}
\def\chpcolortxt{RoyalBlue}
\def\sectionfont{\sffamily\LARGE}

\setcounter{secnumdepth}{10}
\newlength{\hwProblemIndent}
\newlength{\hwProblemWidth}
\newlength{\hwPartIndent}
\newlength{\hwPartWidth}

% Margins
\topmargin=-0.45in
\evensidemargin=0in
\oddsidemargin=0in
\textwidth=6.5in
\textheight=9.0in
\headsep=5pt

%Setup section title and margin
\setlength{\hwProblemIndent}{20pt}
\setlength{\hwProblemWidth}{\textwidth}
\setlength{\hwPartIndent}{20pt}
\setlength{\hwPartWidth}{\textwidth}
\addtolength{\hwProblemWidth}{-\hwProblemIndent}
\addtolength{\hwPartWidth}{-5em}
%Section redifining
%\titleformat{\section}[block]{\bfseries}{\thesection.}{1em}{}
\titleformat{\subsection}[block]{\bfseries}{\thesubsection}{1em}{}
\titleformat{\subsubsection}[block]{}{\thesubsubsection}{1em}{}
\titlespacing*{\subsection} {0em}{3.25ex plus 1ex minus .2ex}{1.5ex plus .2ex}
\titlespacing*{\subsubsection} {3em}{3.25ex plus 1ex minus .2ex}{1.5ex plus .2ex}

%Colorful Section
\newcommand{\hsectionstrut}{\rule{0pt}{12.5pt}}
\newcommand{\makesectionhead}[2]{%
  {\par\vspace{20pt}%
   \parindent 0pt\raggedleft\sectionfont
   \colorbox{\chpcolor}{%
     \parbox[t]{50pt}{\color{white}\hsectionstrut\hfill#2}%
   }%
   \begin{minipage}[t]{\dimexpr\textwidth-50pt-2\fboxsep\relax}
   \color{\chpcolortxt}\hsectionstrut\hspace{5pt}#1
   \end{minipage} \\
   \vspace{10pt}%
  }
}
\newcommand\customfont[1]{{ #1 }}


\linespread{1.1} % Line spacing

% Set up the header and footer
\pagestyle{fancy}
\lhead{\customfont{\color{\hwMaincolor}\hmwkAuthorName}} % Top left header
\chead{\customfont{\color{\hwMaincolor}\hmwkTitle}} % Top center head
\rhead{\customfont{\color{\hwMaincolor}\firstxmark}}% Top right header
\lfoot{\customfont{\color{\hwMaincolor}\lastxmark}} % Bottom left footer
\cfoot{} % Bottom center footer
\rfoot{\customfont{\color{\hwMaincolor}Page\ \thepage\ of\ \protect\pageref{LastPage}}} % Bottom right footer
\renewcommand\headrulewidth{0.4pt} % Size of the header rule
\renewcommand\footrulewidth{0.4pt} % Size of the footer rule
\makeatletter
\renewcommand\headrule{\color{\hwRulecolor} \if@fancyplain \let \headrulewidth \plainheadrulewidth \fi \hrule \@height \headrulewidth \@width \headwidth \vskip \headrulewidth }
\renewcommand\footrule{\color{\hwRulecolor} \if@fancyplain \let \footrulewidth \plainfootrulewidth \fi \vskip -\footruleskip \vskip \footrulewidth \hrule \@width \headwidth \@height \footrulewidth \vskip \footruleskip }

\makeatother

\setlength\parindent{0pt} % Removes all indentation from paragraphs

%----------------------------------------------------------------------------------------
%	CODE INCLUSION CONFIGURATION
%----------------------------------------------------------------------------------------
\definecolor{mygreen}{RGB}{28,172,0} % color values Red, Green, Blue
\definecolor{mylilas}{RGB}{170,55,241}
\lstset{language=Matlab,%
    basicstyle=\footnotesize,
    breaklines=true,%
    morekeywords={matlab2tikz},
    keywordstyle=\color{blue},%
    morekeywords=[2]{1}, keywordstyle=[2]{\color{black}},
    identifierstyle=\color{black},%
    stringstyle=\color{mylilas},
    commentstyle=\color{mygreen},%
    showstringspaces=false,%without this there will be a symbol in the places where there is a space
    numbers=left,%
    numberstyle={\tiny \color{black}},% size of the numbers
    numbersep=9pt, % this defines how far the numbers are from the text
    emph=[1]{for,end,break},emphstyle=[1]\color{red}, %some words to emphasise
    %emph=[2]{word1,word2}, emphstyle=[2]{style},    
    xleftmargin=33pt,
}

%----------------------------------------------------------------------------------------
%	DOCUMENT STRUCTURE COMMANDS
%	Skip this unless you know what you're doing
%----------------------------------------------------------------------------------------

% Header and footer for when a page split occurs within a problem environment
\newcommand{\enterProblemHeader}[1]{
\nobreak\extramarks{#1}{#1 continued on next page\ldots}\nobreak
\nobreak\extramarks{#1 (continued)}{#1 continued on next page\ldots}\nobreak
}

% Header and footer for when a page split occurs between problem environments
\newcommand{\exitProblemHeader}[1]{
\nobreak\extramarks{#1 (continued)}{#1 continued on next page\ldots}\nobreak
\nobreak\extramarks{#1}{}\nobreak
}


\newcommand{\homeworkProblemName}{}
\newenvironment{homeworkProblem}[2]{{
\renewcommand{\homeworkProblemName}{#1 Part} % Assign \homeworkProblemName the name of the problem
\makesectionhead{#2}{#1}% Make a section in the document with the custom problem count
\enterProblemHeader{\homeworkProblemName} % Header and footer within the environment
}}{
\exitProblemHeader{\homeworkProblemName} % Header and footer after the environment
}

\newcommand{\homeworkSectionName}{}
\newenvironment{homeworkSection}[1]{ % New environment for sections within homework problems, takes 1 argument - the name of the section
\renewcommand{\homeworkSectionName}{#1} % Assign \homeworkSectionName to the name of the section from the environment argument
\subsection{\homeworkSectionName} % Make a subsection with the custom name of the subsection
\enterProblemHeader{\homeworkProblemName\ [\homeworkSectionName]} % Header and footer within the environment
}{
\enterProblemHeader{\homeworkProblemName} % Header and footer after the environment
}



\newcommand{\bijooImage}[3][0.8]{
\begin{minipage}{\linewidth}
\includegraphics[width=#1\columnwidth]{Resources/#2}
\centering
\captionof{figure}{#3}
\end{minipage}~\\
}

\newcommand{\bijooDImage}[3][0.8]{
\begin{minipage}{#1\linewidth}
\includegraphics[width=0.8\columnwidth]{Resources/#2.jpg}
\captionsetup{width=0.8\textwidth}
\centering
\captionof{figure}{#3}
\end{minipage}
}
\hypersetup{
    bookmarks=true,         % show bookmarks bar?
    unicode=false,          % non-Latin characters in Acrobat’s bookmarks
    pdftoolbar=true,        % show Acrobat’s toolbar?
    pdfmenubar=true,        % show Acrobat’s menu?
    pdffitwindow=false,     % window fit to page when opened
    pdfstartview={FitH},    % fits the width of the page to the window
    pdftitle={My title},    % title
    pdfauthor={Author},     % author
    pdfsubject={Subject},   % subject of the document
    pdfcreator={Creator},   % creator of the document
    pdfproducer={Producer}, % producer of the document
    pdfkeywords={keyword1, key2, key3}, % list of keywords
    pdfnewwindow=true,      % links in new PDF window
    colorlinks=true,       % false: boxed links; true: colored links
    linkcolor=Black,          % color of internal links (change box color with linkbordercolor)
    citecolor=Black,        % color of links to bibliography
    filecolor=Black,      % color of file links
    urlcolor=Linkblue           % color of external links
}

%
%Listing Format
%

\DeclareCaptionFont{white}{ \color{white} }
\DeclareCaptionFormat{listing}{
  \colorbox[cmyk]{0.43, 0.35, 0.35,0.01 }{
    \parbox{\textwidth}{\hspace{15pt}#1#2#3}
  }
}
\captionsetup[lstlisting]{ format=listing, labelfont=white, textfont=white, singlelinecheck=false, margin=0pt, font={bf,footnotesize} }

      %
     % %
    % % %
   %  %  %
  %       %
 %    %    %
%%%%%%%%%%%%%
%\setcounter{section}{1}
%\setcounter{secnumdepth}{3} % Removes default section numbers

%
%Bijan Codes
%

\newcommand{\bsection}[1]{
  \begin{homeworkProblem}{
        \ifcase\value{section}%
        1st\or%
        2nd\or%
        3rd\or%
        4th\or%
        5th\else%
        6th\fi%
  }{#1}
  \addtocounter{section}{1}
  \setcounter{subsection}{0}
  \setcounter{equation}{0}
  \addcontentsline{toc}{section}{\protect\numberline{\thesection}#1}
\vspace{-1.5em}
}


%----------------------------------------------------------------------------------------
%	NAME AND CLASS SECTION
%----------------------------------------------------------------------------------------

\newcommand{\hmwkTitle}{CA \#1} % Assignment title
\newcommand{\hmwkDueDate}{ Friday, 19 Farvardin 1395 } % Due date
\newcommand{\hmwkClass}{Linear Control Systems} % Course/class
\newcommand{\hmwkClassTime}{} % Class/lecture time
\newcommand{\hmwkClassInstructor}{Dr. Aras Adhami} % Teacher/lecturer
\newcommand{\hmwkAuthorName}{Bijan Binaee} % Your name
\newcommand{\hmwkStudentNum}{810192556} % Your student number

%----------------------------------------------------------------------------------------
%	TITLE PAGE
%----------------------------------------------------------------------------------------

\title{
\vspace{2in}
\textmd{\textbf{\hmwkClass:\ \hmwkTitle}}\\
\normalsize\vspace{0.1in}\small{Due\ on\ \hmwkDueDate}\\
\vspace{0.1in}\large{\textit{\hmwkClassInstructor\ \hmwkClassTime}}
\vspace{3in}
}

\author{\textbf{\hmwkAuthorName} \\{\hmwkStudentNum}}
\date{} % Insert date here if you want it to appear below your name

\renewcommand{\lstlistingname}{Matlab Codes}

%----------------------------------------------------------------------------------------

\begin{document}
\pagecolor{white}
\maketitle
\thispagestyle{empty}
\setcounter{page}{0}
\newpage


%----------------------------------------------------------------------------------------
%	Modeling
%----------------------------------------------------------------------------------------

% To have just one problem per page, simply put a \clearpage after each problem

\begin{homeworkProblem}{1st}{Modeling}
\vspace{-1.5em}

\subsection{State-Space Representation}

\begin{equation}
X \triangleq \begin{bmatrix}
x_1(t)\\
x_2(t)\\
x_3(t)
\end{bmatrix} = \begin{bmatrix}
z(t)\\
\dot{z}(t)\\
i(t)
\end{bmatrix}
\end{equation}
\\Rewriting dynamic equation

\begin{equation}
R~i(t) + \frac{2 \gamma}{z(t)+\beta} \dot{i}(t) - \frac{2 \gamma}{(z(t) + \beta)^2} i(t) ~\dot{z}(t) = u(t) 
\end{equation}

\begin{equation}
m \ddot{z}(t) = m~g - \gamma (\frac{i(t)}{z(t)+\beta})^2
\end{equation}
\\by inspecting Eq. (1.2) and Eq. (1.3) State-Equation are

\begin{equation}
\begin{split}
&\dot{x}_1 (t) = x_2 (t)\\
&\dot{x}_2 (t) = g - \frac{\gamma}{m}(\frac{x_3(t)}{x_1(t)+\beta})^2\\
&\dot{x}_3 (t) = \frac{x_3(t)~x_2(t)}{x_1(t) + \beta}  + (u(t) - R~x_3(t))\frac{x_1(t) + \beta}{2\gamma}
\end{split}
\end{equation}
\\\subsection{Linierazation}
From Jacobian linearization
  \begin{equation}
    A = \frac{\partial f}{\partial x} = 
    \begin{bmatrix}
      0                                                                           & 1                            & 0\\
      \frac{2 \gamma}{m}\frac{(x_3(t))^2}{(x_1(t)+\beta)^3}                       & 0                            & \frac{- 2 \gamma}{m}\frac{x_3(t)}{(x_1(t)+\beta)^2}\\
      \frac{-x_3(t) x_2(t)}{(x_1(t) + \beta)^2} +\frac{u(t) - R~x_3(t)}{2\gamma}  & \frac{x_3(t)}{x_1(t)+\beta}  &  \frac{x_2(t)}{x_1(t) + \beta} - \frac{R(x_1(t) + \beta)}{2\gamma}
    \end{bmatrix}
  \end{equation}
  \begin{equation}
      B = \frac{\partial f}{\partial u} = \begin{bmatrix}
      0                                                     \\
      0                                                     \\
      \frac{x_1(t) + \beta}{2\gamma}
      \end{bmatrix}u(t)
  \end{equation}
\\and\\
    \begin{equation}
      \begin{gathered}
        x_e = \begin{bmatrix}
        0.002\\
        0\\
        0.8391
        \end{bmatrix}~,~ u_e = R ~ i(t) = 8.391 \\
        \beta = 2 \times 10^{-5} ~ , ~ \gamma = 1.131 \times 10^{-5} ~ , ~ m = 0.2
      \end{gathered}
    \end{equation}
\\substituing Eq. (1.7) in Eq. (1.6) and Eq (1.5)\\
  \begin{equation}
    A = 
    \begin{bmatrix}
      0       & 1         & 0                                         \\
      9661.31 & 0         & -23.2581                                  \\
      0       & 415.396   & -893.015
    \end{bmatrix}
    ~,~
    B = 
    \begin{bmatrix}
      0\\
      0\\
      89.3015
    \end{bmatrix}
  \end{equation}
In vector-matrix form,\\
  \begin{equation}
    \begin{bmatrix}
      \dot{x}_1(t)\\
      \dot{x}_2(t)\\
      \dot{x}_3(t)
    \end{bmatrix} 
    = 
    \begin{bmatrix}
      0       & 1         & 0                                         \\
      9661.31 & 0         & -23.2581                                  \\
      0       & 415.396   & -893.015
    \end{bmatrix}
    \begin{bmatrix}
      x_1(t)\\
      x_2(t)\\
      x_3(t)
    \end{bmatrix}
    +
    \begin{bmatrix}
      0\\
      0\\
      89.3015
    \end{bmatrix}
    u(t)
  \end{equation}
  

\end{homeworkProblem}



%----------------------------------------------------------------------------------------
%	Simulation
%----------------------------------------------------------------------------------------




\begin{homeworkProblem}{2nd}{Simulation}
\addtocounter{section}{1}
\setcounter{subsection}{0}
\setcounter{equation}{0}


\subsection{Eigen Values}
From matlab \textit{eig} command eigen vector,\\
  \begin{equation}
    e =
    \begin{bmatrix}
       93.5172\\
     -104.6099\\
     -881.9214\\
    \end{bmatrix}
  \end{equation}
\\~\\
\\Eq (2.1) imply a positive pole of system which cause an unstable behavior due to specific inputs. In the Active Magnetic Bearing, this pole, cause oscillating in bearing position.

\subsection{Transfer Function}
From control theory
  \begin{equation}
    H(s) = \mathbf{C} ~ (s\mathbf{I}-\mathbf{A})^{-1} ~ \mathbf{B}
  \end{equation}
In the case \\
  \begin{equation}
    y =     
    \begin{bmatrix}
      1    &  0   &   0
    \end{bmatrix} 
    X
  \end{equation}
\\then\\
  \begin{equation}
    C =     
    \begin{bmatrix}
      1    &  0   &   0
    \end{bmatrix} 
    ~ , ~ 
    D = 0
  \end{equation}
\\Using matlab \textit{ss2tf} \\
  \begin{equation}
    H(s) = \frac{-2077}{s^3+893.0140 s ^ 2 + 0.0117 s + 8.627 \times 10^6}
  \end{equation}
\\employing matlab \textit{tf2ss} command, the state equation is\\
  \begin{equation}
    \begin{gathered}
      A = 
      \begin{bmatrix}
        -893.0140  & -0.0117   & 8.627 \times 10^6                                         \\
        1          & 0         & 0                                 \\
        0          & 1         & 0
      \end{bmatrix}
      ~,~
      B = 
      \begin{bmatrix}
        1\\
        0\\
        0
      \end{bmatrix}
      \\~\\
      C = 
      \begin{bmatrix}
        0       & 0         & -2077                                         \\
      \end{bmatrix}
      ~,~
      D = 0
    \end{gathered}
  \end{equation}
  Comparing Eq. (2.6) and Eq. (1.9) the state equation is changed which imply that state variables are vary to choose from.
  
\subsection{Systems Pole}
  Input
\lstinputlisting[caption=Frequency Response]{Codes/ca1.m}
    \begin{equation}
      H(s) = \mathbf{C} ~ (s\mathbf{I}-\mathbf{A})^{-1} ~ \mathbf{B}
    \end{equation}
  In the case \\
    \begin{equation}
      y =     
      \begin{bmatrix}
        1    &  0   &   0
      \end{bmatrix} 
      X
    \end{equation}
  \\then\\
    \begin{equation}
      C =     
      \begin{bmatrix}
        1    &  0   &   0
      \end{bmatrix} 
      ~ , ~ 
      D = 0
    \end{equation}
  \\Using matlab \textit{ss2tf} \\
    \begin{equation}
      H(s) = \frac{-2077}{s^3+893.0140 s ^ 2 + 0.0117 s + 8.627 \times 10^6}
    \end{equation}
  \\employing matlab \textit{tf2ss} command, the state equation is\\
    \begin{equation}
      \begin{gathered}
        A = 
        \begin{bmatrix}
          -893.0140  & -0.0117   & 8.627 \times 10^6                                         \\
          1          & 0         & 0                                 \\
          0          & 1         & 0
        \end{bmatrix}
        ~,~
        B = 
        \begin{bmatrix}
          1\\
          0\\
          0
        \end{bmatrix}
        \\~\\
        C = 
        \begin{bmatrix}
          0       & 0         & -2077                                         \\
        \end{bmatrix}
        ~,~
        D = 0
      \end{gathered}
    \end{equation}
    Comparing Eq. (2.6) and Eq. (1.9) the state equation is changed which imply that state variables are vary to choose from.
  
  
  
  
  
  
\end{homeworkProblem}




%----------------------------------------------------------------------------------------
%	PROBLEM 2
%----------------------------------------------------------------------------------------
\newpage
\begin{homeworkProblem}{2nd}{Design}
\vspace{-1.5em}
\addtocounter{section}{1}
\setcounter{subsection}{0}
\subsection{‫‪Constants}
To calculate parameter, $\lambda_g$ is required. from microwave engineering  
$$\beta = \sqrt{k^2 - \frac{m \pi}{a} ^ 2 - \frac{n \pi}{b} ^ 2}$$ 
$$ a = 3.75 mm ~~ , ~~ b = 20.06 mm  ~~ , ~~ f = 55.6 GHz $$
thus\\
$$ \lambda_g = \frac{2 \pi}{\beta} = 7.73 mm $$
\subsection{Parameters Calculation}
Remark from the previous section 
$$ l = \frac{\lambda}{4} = 1.93 mm $$
$$ \text{offset} = \sqrt{2} ~ \frac{\lambda}{4} = 2.72 mm $$
$$ w = \frac{l}{7} = 0.3 mm $$
\end{homeworkProblem}
%----------------------------------------------------------------------------------------
%	PROBLEM 3
%----------------------------------------------------------------------------------------

\begin{homeworkProblem}{3rd}{Results}
\vspace{-1.5em}
\addtocounter{section}{1}
\setcounter{subsection}{0}
\subsection{‫‪Optimization}
By sweep on on w, offset and l to attain -20dB coupling factor and maximum bandwidth following parameter are picked up.\\
$$ l = 1.6 ~~ , ~~ w = 0.4 ~~ , ~~  \text{offset} = 2.72 $$
A sample sweep on parameter "W" shown in Fig 2. The results shown as W increases the coupling factor increases but it affect decoupling factor in a harmful way.\\
\begin{minipage}{\linewidth}
\includegraphics[width=0.8\columnwidth]{Resources/w}
\centering
\captionof{figure}{power transfer to each port from port 1 over 300u to 500u on "W"}
~\\
\end{minipage}
\subsection{Final Structure}
\begin{minipage}{\linewidth}
\includegraphics[width=0.8\columnwidth]{Resources/view}
\centering
\captionof{figure}{proposed structure in HFSS design view}
~\\
\end{minipage}
\begin{minipage}{\linewidth}
\includegraphics[width=0.8\columnwidth]{Resources/side}
\centering
\captionof{figure}{proposed structure from the side view}
~\\
\end{minipage}
\begin{minipage}{\linewidth}
\includegraphics[width=0.8\columnwidth]{Resources/up}
\centering
\captionof{figure}{proposed structure from the top view}
~\\
\end{minipage}
\subsection{Frequency Sweep}
\begin{minipage}{\linewidth}
\includegraphics[width=0.8\columnwidth]{Resources/BW}
\centering
\captionof{figure}{power transfer on frequency 53GHz to 58GHz }
~\\
\end{minipage}
\subsection{Fields Plot}
\begin{minipage}{\linewidth}
\includegraphics[width=0.8\columnwidth]{Resources/Fields}
\centering
\captionof{figure}{Ridge‬‬ ‫‪Waveguide‬‬ Imaginary propagation constant frequency sweep for $TE_{01}$ and $TE_{20}$}
~\\
\subsection{Results}
$$ l = 1.6 ~~ , ~~ w = 0.4 ~~ , ~~  \text{offset} = 2.72 ~~ , ~~ f_o = 55.6GHz $$
$$ \text{Coupling Factor} = -20dB ~~ , ~~ \text{deCoupling Factor} < -52dB ~~ , ~~  \text{Minimum Bandwidth} = 5GHz $$
\end{minipage}
\end{homeworkProblem}
\end{document}
