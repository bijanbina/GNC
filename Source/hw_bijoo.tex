%equation contains section
\numberwithin{equation}{section}  

%PineGreen
\definecolor{ao(english)}{rgb}{0.0, 0.5, 0.0}
\definecolor{airforceblue}{rgb}{0.27, 0.47, 0.6}
\definecolor{Headblue}{rgb}{0.25, 0.37, 0.5}
\definecolor{Linkblue}{rgb}{0.20, 0.4, 0.73}
\newcommand{\hwMaincolor}{black}
\newcommand{\hwRulecolor}{black}
\def\chpcolor{airforceblue}
\def\chpcolortxt{RoyalBlue}
\def\sectionfont{\sffamily\LARGE}

\setcounter{secnumdepth}{10}
\newlength{\hwProblemIndent}
\newlength{\hwProblemWidth}
\newlength{\hwPartIndent}
\newlength{\hwPartWidth}

% Margins
\topmargin=-0.45in
\evensidemargin=0in
\oddsidemargin=0in
\textwidth=6.5in
\textheight=9.0in
\headsep=5pt

%Setup section title and margin
\setlength{\hwProblemIndent}{20pt}
\setlength{\hwProblemWidth}{\textwidth}
\setlength{\hwPartIndent}{20pt}
\setlength{\hwPartWidth}{\textwidth}
\addtolength{\hwProblemWidth}{-\hwProblemIndent}
\addtolength{\hwPartWidth}{-5em}
%Section redifining
%\titleformat{\section}[block]{\bfseries}{\thesection.}{1em}{}
\titleformat{\subsection}[block]{\bfseries}{\thesubsection}{1em}{}
\titleformat{\subsubsection}[block]{}{\thesubsubsection}{1em}{}
\titlespacing*{\subsection} {0em}{3.25ex plus 1ex minus .2ex}{1.5ex plus .2ex}
\titlespacing*{\subsubsection} {3em}{3.25ex plus 1ex minus .2ex}{1.5ex plus .2ex}

%Colorful Section
\newcommand{\hsectionstrut}{\rule{0pt}{12.5pt}}
\newcommand{\makesectionhead}[2]{%
  {\par\vspace{20pt}%
   \parindent 0pt\raggedleft\sectionfont
   \colorbox{\chpcolor}{%
     \parbox[t]{50pt}{\color{white}\hsectionstrut\hfill#2}%
   }%
   \begin{minipage}[t]{\dimexpr\textwidth-50pt-2\fboxsep\relax}
   \color{\chpcolortxt}\hsectionstrut\hspace{5pt}#1
   \end{minipage} \\
   \vspace{10pt}%
  }
}
\newcommand\customfont[1]{{ #1 }}


\linespread{1.1} % Line spacing

% Set up the header and footer
\pagestyle{fancy}
\lhead{\customfont{\color{\hwMaincolor}\hmwkAuthorName}} % Top left header
\chead{\customfont{\color{\hwMaincolor}\hmwkTitle}} % Top center head
\rhead{\customfont{\color{\hwMaincolor}\firstxmark}}% Top right header
\lfoot{\customfont{\color{\hwMaincolor}\lastxmark}} % Bottom left footer
\cfoot{} % Bottom center footer
\rfoot{\customfont{\color{\hwMaincolor}Page\ \thepage\ of\ \protect\pageref{LastPage}}} % Bottom right footer
\renewcommand\headrulewidth{0.4pt} % Size of the header rule
\renewcommand\footrulewidth{0.4pt} % Size of the footer rule
\makeatletter
\renewcommand\headrule{\color{\hwRulecolor} \if@fancyplain \let \headrulewidth \plainheadrulewidth \fi \hrule \@height \headrulewidth \@width \headwidth \vskip \headrulewidth }
\renewcommand\footrule{\color{\hwRulecolor} \if@fancyplain \let \footrulewidth \plainfootrulewidth \fi \vskip -\footruleskip \vskip \footrulewidth \hrule \@width \headwidth \@height \footrulewidth \vskip \footruleskip }

\makeatother

\setlength\parindent{0pt} % Removes all indentation from paragraphs

%----------------------------------------------------------------------------------------
%	CODE INCLUSION CONFIGURATION
%----------------------------------------------------------------------------------------
\definecolor{mygreen}{RGB}{28,172,0} % color values Red, Green, Blue
\definecolor{mylilas}{RGB}{170,55,241}
\lstset{language=Matlab,%
    basicstyle=\footnotesize,
    breaklines=true,%
    morekeywords={matlab2tikz},
    keywordstyle=\color{blue},%
    morekeywords=[2]{1}, keywordstyle=[2]{\color{black}},
    identifierstyle=\color{black},%
    stringstyle=\color{mylilas},
    commentstyle=\color{mygreen},%
    showstringspaces=false,%without this there will be a symbol in the places where there is a space
    numbers=left,%
    numberstyle={\tiny \color{black}},% size of the numbers
    numbersep=9pt, % this defines how far the numbers are from the text
    emph=[1]{for,end,break},emphstyle=[1]\color{red}, %some words to emphasise
    %emph=[2]{word1,word2}, emphstyle=[2]{style},    
    xleftmargin=33pt,
}

%----------------------------------------------------------------------------------------
%	DOCUMENT STRUCTURE COMMANDS
%	Skip this unless you know what you're doing
%----------------------------------------------------------------------------------------

% Header and footer for when a page split occurs within a problem environment
\newcommand{\enterProblemHeader}[1]{
\nobreak\extramarks{#1}{#1 continued on next page\ldots}\nobreak
\nobreak\extramarks{#1 (continued)}{#1 continued on next page\ldots}\nobreak
}

% Header and footer for when a page split occurs between problem environments
\newcommand{\exitProblemHeader}[1]{
\nobreak\extramarks{#1 (continued)}{#1 continued on next page\ldots}\nobreak
\nobreak\extramarks{#1}{}\nobreak
}


\newcommand{\homeworkProblemName}{}
\newenvironment{homeworkProblem}[2]{{
\renewcommand{\homeworkProblemName}{#1 Part} % Assign \homeworkProblemName the name of the problem
\makesectionhead{#2}{#1}% Make a section in the document with the custom problem count
\enterProblemHeader{\homeworkProblemName} % Header and footer within the environment
}}{
\exitProblemHeader{\homeworkProblemName} % Header and footer after the environment
}

\newcommand{\homeworkSectionName}{}
\newenvironment{homeworkSection}[1]{ % New environment for sections within homework problems, takes 1 argument - the name of the section
\renewcommand{\homeworkSectionName}{#1} % Assign \homeworkSectionName to the name of the section from the environment argument
\subsection{\homeworkSectionName} % Make a subsection with the custom name of the subsection
\enterProblemHeader{\homeworkProblemName\ [\homeworkSectionName]} % Header and footer within the environment
}{
\enterProblemHeader{\homeworkProblemName} % Header and footer after the environment
}



\newcommand{\bijooImage}[3][0.8]{
\begin{minipage}{\linewidth}
\includegraphics[width=#1\columnwidth]{Resources/#2}
\centering
\captionof{figure}{#3}
\end{minipage}~\\
}

\newcommand{\bijooDImage}[3][0.8]{
\begin{minipage}{#1\linewidth}
\includegraphics[width=0.8\columnwidth]{Resources/#2.jpg}
\captionsetup{width=0.8\textwidth}
\centering
\captionof{figure}{#3}
\end{minipage}
}
\hypersetup{
    bookmarks=true,         % show bookmarks bar?
    unicode=false,          % non-Latin characters in Acrobat’s bookmarks
    pdftoolbar=true,        % show Acrobat’s toolbar?
    pdfmenubar=true,        % show Acrobat’s menu?
    pdffitwindow=false,     % window fit to page when opened
    pdfstartview={FitH},    % fits the width of the page to the window
    pdftitle={My title},    % title
    pdfauthor={Author},     % author
    pdfsubject={Subject},   % subject of the document
    pdfcreator={Creator},   % creator of the document
    pdfproducer={Producer}, % producer of the document
    pdfkeywords={keyword1, key2, key3}, % list of keywords
    pdfnewwindow=true,      % links in new PDF window
    colorlinks=true,       % false: boxed links; true: colored links
    linkcolor=Black,          % color of internal links (change box color with linkbordercolor)
    citecolor=Black,        % color of links to bibliography
    filecolor=Black,      % color of file links
    urlcolor=Linkblue           % color of external links
}

%
%Listing Format
%

\DeclareCaptionFont{white}{ \color{white} }
\DeclareCaptionFormat{listing}{
  \colorbox[cmyk]{0.43, 0.35, 0.35,0.01 }{
    \parbox{\textwidth}{\hspace{15pt}#1#2#3}
  }
}
\captionsetup[lstlisting]{ format=listing, labelfont=white, textfont=white, singlelinecheck=false, margin=0pt, font={bf,footnotesize} }

      %
     % %
    % % %
   %  %  %
  %       %
 %    %    %
%%%%%%%%%%%%%
%\setcounter{section}{1}
%\setcounter{secnumdepth}{3} % Removes default section numbers

%
%Bijan Codes
%

\newcommand{\bsection}[1]{
  \begin{homeworkProblem}{
        \ifcase\value{section}%
        1st\or%
        2nd\or%
        3rd\or%
        4th\or%
        5th\else%
        6th\fi%
  }{#1}
  \addtocounter{section}{1}
  \setcounter{subsection}{0}
  \setcounter{equation}{0}
  \addcontentsline{toc}{section}{\protect\numberline{\thesection}#1}
\vspace{-1.5em}
}
